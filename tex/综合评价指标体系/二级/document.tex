%%  通过自定义一级指标间隔角度(sibling angle)与初始角度(clockwise from)改变一级拓扑结构
%%  通过自定义二级指标间隔角度与初始角度改变二级拓扑结构
%%  必要情况可以给节点增加超链接
\documentclass[UTF8]{ctexart}
\usepackage{fullpage}
\textheight=9.0in
\pagestyle{empty}
\raggedbottom
\raggedright
\setlength{\tabcolsep}{0in}
\usepackage[utf8]{inputenc}
\usepackage{dtklogos}
\usepackage{tikz}
\usepackage{dtklogos}
\usetikzlibrary{mindmap,shadows}
\usepackage[hidelinks,pdfencoding=auto]{hyperref}
% Information boxes
\newcommand*{\info}[4][16.3]{%
	\node [ annotation, #3, scale=0.65, text width = #1em,
	inner sep = 2mm ] at (#2) {%
		\list{$\bullet$}{\topsep=0pt\itemsep=0pt\parsep=0pt
			\parskip=0pt\labelwidth=8pt\leftmargin=8pt
			\itemindent=0pt\labelsep=2pt}%
		#4
		\endlist
	};
}
\usepackage{verbatim}
\usepackage{smartdiagram}
\begin{document}
\begin{center}
\centering

\begin{tikzpicture}[ every annotation/.style = {draw,
	fill = white, font = \Large}]
\path[mindmap,concept color=black!40,text=white,
every node/.style={concept,circular drop shadow},
root/.style    = {concept color=black!40,
	font=\large\bfseries,text width=10em},
level 1 concept/.append style={font=\Large\bfseries,
	sibling angle=90,text width=7.7em,  % 间隔角度等参数
	level distance=15em,inner sep=0pt},
level 2 concept/.append style={font=\bfseries,level distance=9em,sibling angle=90},
]
node[root] {长江水质综合评价体系} [clockwise from=45]
child[concept color=blue!60] {
	node {\href{http://golatex.de}{一级指标2}} [clockwise from=135] %% 初始节点角度参数
	child { node (goForum) {{二级指标2.1}} }
	child { node (goWiki) {{二级指标2.2}} }
	child { node (goWiki) {{二级指标2.3}} }
}
 child[concept color=red] {
	node[concept] (PGFPlots) {{一级指标3}}
	[clockwise from=45]
	child { node (goForum) {{二级指标2.1}} }
	child { node (goWiki) {{二级指标2.2}} }
	child { node (goWiki) {{二级指标2.3}} }
}
child[concept color=green!40!black] {
	node[concept] {{一级指标4}}
	[clockwise from=-45]
	child { node[concept] (TikZGalerie) 
		{{二级指标4.1}} }
	child { node[concept] (TeXampleBlog)
		{{二级指标4.2}} }
	child { node[concept] (Planet)
		{{二级指标4.3}} }
}
child[concept color=yellow!60!black] {
	node[concept] (Blogs) {一级指标1} [clockwise from=225]
	child { node[concept] {{二级指标1.1}}}
	child { node[concept] {{二级指标1.2}} }
	child { node[concept] (Cookbook)
		{{二级指标1.3}}}
};
\end{tikzpicture}
\end{center}
\end{document}