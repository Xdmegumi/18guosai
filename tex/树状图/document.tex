% Decision tree
% Author: Stefan Kottwitz
% https://www.packtpub.com/hardware-and-creative/latex-cookbook
\documentclass[UTF8,border=18pt]{ctexart}
%%%<
\usepackage{verbatim}
%%%>
\begin{comment}
:Title: Decision tree
:Tags: Trees;Cookbook
:Author: Stefan Kottwitz
:Slug: decision-tree

A horizontal tree, growing to the right.
I created a basic style for tree nodes, and
derived styles for specific kinds of nodes.
\end{comment}
\usepackage{tikz}
\tikzset{
	treenode/.style = {shape=rectangle, rounded corners,
		draw, align=center,
		top color=white, bottom color=blue!20},
	bottom/.style     = {treenode, font=\Large, bottom color=red!30},
	root/.style     = {treenode, font=\Large, bottom color=red!10},
	env/.style      = {treenode, font=\ttfamily\normalsize},
	dummy/.style    = {circle,draw}
}
\begin{document}
	
	\begin{tikzpicture}
	[
	grow                    =-90,  % 生长方向
	%    start =going below,
	% sibling distance        = 3cm,  % 层内距离
	% level distance          = 8em,  % 层间距离
	level 1/.style={sibling distance=8cm,level distance=3.5cm},  %% 各层级距离
	level 2/.style={sibling distance=3.2cm,level distance=3.5cm},
	level 3/.style={sibling distance=2cm,level distance=3.5cm},
	level 4/.style={sibling distance=1cm,level distance=3.5cm},
	edge from parent/.style = {draw, -latex},
	every node/.style       = {font=\normalsize},
	sloped
	]
	\node [bottom] {高等数学学生评价指标体系 $S$}
	child { node [root] {过程考核 $w_1S_1$} % 空节点
		child { node [env] {$S_1=\sum \limits_{i=1}^4a_iX^{(i)}$}
			edge from parent node [below] {问题 2} }
		edge from parent node [above] {考察性评价} }
	child { node [root] {过程控制 $w_2S_2$} % 空节点
		child { node [env] {$s_1$:课堂主动回答}
			edge from parent node [above, align=center]
			{$c_1$}}
		child { node [env] {$s_2$:课后交流}
			edge from parent node [above, align=center]
			{$c_2$}}
    	child { node [env] {$s_3$:小组讨论}
			edge from parent node [above, align=center]
			{$c_3$}}
		edge from parent node [above] {表现性评价} };
	\end{tikzpicture}
	
\end{document}